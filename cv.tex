
\PassOptionsToPackage{dvipsnames}{xcolor}


%% Use the "normalphoto" option if you want a normal photo instead of cropped to a circle
%\documentclass[10pt,a4paper,normalphoto]{altacv}

\documentclass[10pt,a4paper,ragged2e,withhyper]{cv}

\geometry{left=1.2cm,right=1.2cm,top=1cm,bottom=1cm,columnsep=0.75cm}
\usepackage{paracol}


\ifxetexorluatex
  \setmainfont{Roboto Slab}
  \setsansfont{Lato}
  \renewcommand{\familydefault}{\sfdefault}
\else
  \usepackage[rm]{roboto}
  \usepackage[defaultsans]{lato}
  \renewcommand{\familydefault}{\sfdefault}
\fi

% ----- LIGHT MODE -----
\definecolor{SlateGrey}{HTML}{2E2E2E}
\definecolor{LightGrey}{HTML}{666666}
\definecolor{PrimaryColor}{HTML}{001F5A}
\definecolor{SecondaryColor}{HTML}{0039AC}
\definecolor{ThirdColor}{HTML}{F3890B}
\definecolor{BackgroundColor}{HTML}{E2E2E2}
\colorlet{name}{PrimaryColor}
\colorlet{tagline}{PrimaryColor}
\colorlet{heading}{PrimaryColor}
\colorlet{headingrule}{ThirdColor}
\colorlet{subheading}{SecondaryColor}
\colorlet{accent}{SecondaryColor}
\colorlet{emphasis}{SlateGrey}
\colorlet{body}{LightGrey}
\pagecolor{BackgroundColor}   
% ----- DARK MODE -----
%\definecolor{BackgroundColor}{HTML}{242424}
%\definecolor{SlateGrey}{HTML}{6F6F6F}
%\definecolor{LightGrey}{HTML}{ABABAB}
%\definecolor{PrimaryColor}{HTML}{3F7FFF}
%\colorlet{name}{PrimaryColor}
%\colorlet{tagline}{PrimaryColor}
%\colorlet{heading}{PrimaryColor}
%\colorlet{headingrule}{PrimaryColor}
%\colorlet{subheading}{PrimaryColor}
%\colorlet{accent}{PrimaryColor}
%\colorlet{emphasis}{LightGrey}
%\colorlet{body}{LightGrey}
%\pagecolor{BackgroundColor}

% Change some fonts, if necessary
\renewcommand{\namefont}{\Huge\rmfamily\bfseries}
\renewcommand{\personalinfofont}{\small\bfseries}
\renewcommand{\cvsectionfont}{\LARGE\rmfamily\bfseries}
\renewcommand{\cvsubsectionfont}{\large\bfseries}

% Change the bullets for itemize and rating marker
\renewcommand{\itemmarker}{{\small\textbullet}}
\renewcommand{\ratingmarker}{\faCircle}

\usepackage[french]{babel}

\begin{document}
    \name{Adrien Thomas}
    \tagline{Élève ingénieur AgroParisTech}
    \photoL{4cm}{photo}
    
    \personalinfo{
        \email{adrien.thomas@agroparistech.fr}\smallskip
        \phone{+33 6 52 31 99 75}
        \location{Paris, France}\\
        %\github{id}\\
        %\homepage{nicolasomar.me}
        %\medium{nicolasomar}
        %% \printinfo{symbol}{detail}[optional hyperlink prefix]
        
        \printinfo{\faAddressCard}{Né le 21 Mai 1998 à Lyon}[]
        % ex autre champ
        % \NewInfoField{gitlab}{\faGitlab}[https://gitlab.com/]
        % \gitlab{your_id}
    }
    
    \makecvheader
    
    %ratio gauche droite
    \columnratio{0.3}

    \begin{paracol}{2}
        
        % ----- COMPETENCES -----
        \cvsection{Compétences}
            \cvtag{PSC1}
            \cvtag{Permis B}
            \cvtag{BAFA}
            
            \cvtag{Certification Voltaire}
            \medskip
            
            \cvtag{Pack Office}
            \cvtag{Python}
            
            \cvtag{R}
            \cvtag{SQL}
            \cvtag{Premiere Pro}
            
            \cvtag{Photoshop}
            \cvtag{LaTeX}
            \cvtag{Linux}
        % ----- COMPETENCES -----
        
        % ----- LANGUES -----
        \cvsection{Langues}
            \cvlang{Français}{Natif}
            \divider
            \cvlang{Anglais}{B2}
            \divider
            \cvlang{Chinois}{Bases}
            
            
        % ----- LANGUES -----
        
        % ----- LOISIRS -----
        \cvsection{Loisirs}
            \cvtag{VTT (FFCT)}
            \cvtag{Trail}
            \cvtag{Escalade}
            
            \cvtag{Plongée}
            \cvtag{Escrime}
            \medskip
            
            \cvtag{Scoutisme (depuis 2007)}
            
            \cvtag{Animateur jeunesse}
            
            \cvtag{Informatique}
        % ----- LOISIRS -----    
        
        % ----- PROJETS -----
        \cvsection{Projets divers}
        
            \cvevent{IGEM }{| MIT \cvrepo{| \faGlobe}{https://igem.org/Main_Page}}{Octobre 2020- Septembre 2021}{}
            \begin{itemize}
                \item Compétition universitaire de biologie de synthèse
                \item Vice-président de l'équipe d'Evry
            \end{itemize}
            \bigbreak
            \cvevent{Projet Solidaire en Bulgarie}{| SGDF \cvrepo{| \faGlobe}{https://www.sgdf.fr/}}{été 2016}{}
            \begin{itemize}
                \item Collecte et gestion d'un budget de 10 000€ 
                \item Gestion de la logistique et de l'intendance sur place pour une équipe de 6 personnes
            \end{itemize}
        % ----- PROJETS -----
        
        \newpage
        
        %On passe à droite
        \switchcolumn
        
        % ----- OBJET (ou description) -----
        \cvsection{Objet}
            \begin{quote}
                Candidature ...
            \end{quote}
        % ----- OBJET (ou description) -----
        
        % ----- EXPERIENCE -----
        \cvsection{Experience}
            %\cvevent{Employé}{| Salaison Torrilhon}{3 moi, Juin 2020 - Aout %2020}{Cheyrac, Haute Loire, France}
            %\begin{itemize}
            %    \item Manipulation et vaccination des porcelets
            %    \item Entretient des zones d'engraissement (paillées)
            %    \item Préparation des commandes et Vente au client
            %\end{itemize}
            %\divider
            
            \cvevent{Stage}{| UMR MIA Paris}{30 jours sur l'année 2020-2021}{Paris, France}
            Comparaison des méthodes d'analyse de données compositionnelles et des approches basées sur des modèles linéaires généralisés pour l'analyse de données métagénomiques
            \smallbreak
            \begin{itemize}
                \item Recherches bibliographiques, état de l'art
                \item Etude de simulations pour comparer les différentes approches
                \item Application(s) en métagénomique 
            \end{itemize}
             \divider
            \cvevent{Stage}{| Institut Terres Inovia}{17 jours sur l'année 2019-2020}{Grignon, France}
            Projet de recherche sur la lutte contre les ravageurs multirésistants du colza
            \smallbreak
            \begin{itemize}
                \item Réorganisation de la base de données
                \item Automatisation de la saisie des valeurs expérimentales
                \item Exploitation des résultats (R)
            \end{itemize}
            
            \divider
            
            \cvevent{Stage}{| UGSF - UMR 8576 CNRS - Université de Lille}{1 semaine, Février 2016}{Lille, France}
            Mécanismes moléculaires de la N-glycosylation et pathologies associées.
            \smallbreak
            \begin{itemize}
                \item Utilisation du modèle levure comme miroir des mécanismes molèculaires humains
                \item Manipulations : transformation et clonage de levures
            \end{itemize}
        % ----- EXPERIENCE -----
        
        % ----- FORMATION -----
        \cvsection{Formation}
            \cvevent{Ecole d'ingénieur}{| AgroParisTech}{2019-2021}{Paris-Grignon, France}
            \begin{itemize}
                \item 2\ieme{} année : Santé et nutrition
                \item 1\ier{} année : Générale
            \end{itemize}
            \divider
            
            \cvevent{Classe préparatoire BCPST}{| Assomption Bellevue}{2016-2019}{Lyon, France}
            \divider
            
            \cvevent{Baccalauréat S mention Bien}{| Lycée Aux Lazaristes}{2016}{Lyon, France}
        % ----- FORMATION -----
        
    \end{paracol}
\end{document}